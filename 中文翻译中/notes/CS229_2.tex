\documentclass[14pt,a4paper]{article}
\usepackage[utf8]{inputenc}
\usepackage{xeCJK}
\setCJKmainfont[BoldFont=SimHei]{SimSun}
\setCJKmonofont{SimSun}% 设置缺省中文字体
\usepackage{amsmath}
\usepackage{amsfonts}
\usepackage{amssymb}
\usepackage{makeidx}
\usepackage{graphicx}
\usepackage{setspace}
\usepackage[left=2cm,right=2cm,top=2cm,bottom=2cm]{geometry}
\title{CS229 Lecture notes}
\author{原作者:Andrew Ng (吴恩达),翻译:CylcerUser} 
\date{}
\begin{document}
\maketitle

\section{因子分析(Factor analysis)}  

\begin{spacing}{1.5}

\quad \quad 如果有一个高斯模型混合(a mixture of several Gaussians)而来的数据集$x^{(i)}\in R^{n}$,那么就可以用期望最大化算法(EM algorithm)来对这个混合模型(mixture model)进行拟合。这种情况下,对于有充足数据(sufficient data)的问题,我们通常假设可以从数据中识别出多个高斯模型结构(multiple-Gaussian structure)。例如,如果我们的训练样本集合规模(training set size)m 远远大于(significantly larger than)数据的维度(dimension)n,就符合这种情况。
然后来考虑一下反过来的情况,也就是$n$远远大于$m$,即$n>m$。在这样的问题中,就可能单独一个高斯模型来对数据建模很艰难,更不用说了高斯模型的混合模型了。由于$m$个数据点所张开(span)的只是一个$n$维空间$R^n$的低维度子空间(low-dimensional subspace),如果用高斯模型(Gaussian)对数据进行建模,然后还是用常规的最大似然估计(usual maximum likelihood estimators)来估计(estimate)平均值(mean)和方差(covariance),得到的则是:
$$
\begin{aligned}
\mu&=\frac{1}{m}\sum_{i=1}^{m}x^{(i)}  \\
\Sigma&=\frac{1}{m}\sum_{i=1}^{m}(x^{(i)}-\mu)(x^{(i)}-\mu)^{T}
\end{aligned}
$$
we would find that the matrix $\Sigma$ is singular.This means that $\Sigma^{-1}$ does not exist,and $1\|\Sigma|^{\frac{1}{2}}=1/0$.But both of these terms are needed in computing the usual density of a multivariate Gaussian distribution.Another way of the stating this difficulty is that maximum likelihood estimate of the parmeters result in a Gaussian that places all of its probability in the affine space spanned by the data \footnote{This is the set of points x satisfying $x=\sum_{i=1}^{m}\alpha_{i}x^{(i)}$,for some $\alpha_{i}$'s so that $\sum_{i=1}^{m}\alpha_{i}=1$}。,and the corresponds to a singular convariance matrix.

我们会发现这里的$\Sigma$是一个奇异(singular)矩阵。这就意味着其逆矩阵$\Sigma$不存在,而$1/|\Sigma|^{\frac{1}{2}}=1/0$。但这几个变量都是必须的,要用来计算一个多元高斯函数分布(multivariate Gaussian distribution)的常规密度函数(usual density)。 还可以用另外一种方法来讲述清楚这个难题,也就是对参数(parameters)的最大似然估计(maximum likelihood:estimates)会产生一个高斯分布(Gausssian),其概率分分布在有样本数据所张成的放射空间(affine space)中,对应着一个奇异的协方差矩阵(singular convariance matrix)。

通常情况下,除非m比n大出相当多(some reasonable amount),否则最大似然估计(maximum likelihood estimates)得到的均值(mean)和方差(convariance)都会很差(quite poor)。 尽管如此,我们还是希望能用自己已有的数据,拟合出一条合理(reasonable)的高斯模型(Gaussian model),而且希望能够识别出数据中的某些有意义的协方差结构(convariance structure)。那这可怎们办?

在接下来的这一部分内容里,我们首先回顾一下对$\Sigma$的两个可能约束(possible restrictions),这来那个约束条件能让我们使用小规模数据来拟合$\Sigma$,但是都不能就我们的问题给出让人满意的解(staisfactory solution)。 然后接下来我们要讨论一下高斯模型的边界和条件分布。最后,我们会讲下因子分析模型(factor analysis model),以及对应的期望最大化算法(EM algorithm)。
\end{spacing}
\subsection {$\Sigma$的约束条件} 
\begin{spacing}{1.5}
\quad \quad 如果我们没有充足的数据来拟合一个完整的协方差矩阵(covariance matrix),就可以对矩阵空间$\Sigma$给出某些约束条件(restrictions)。例如,我们可以选择拟合一个对角(diagonal)的协方差矩阵$\Sigma$。这样,读者很容易就能验证这样的一个协方差矩阵的最大似然估计(maximum lilklihoof estimate),可以有对角矩阵(diagonal matrix)$\Sigma$满足:
$$
\Sigma_{jj}=\frac{1}{m}\Sigma_{i=1}^{m}(x_{j}^{i}-\mu_{j})^2
$$
因此,$\Sigma$就是对数据中的$j$个坐标位置的方差的经验估计(empeirical estimate)。

Recall that the contours of a Gaussian density are ellipses. A diagonal $\Sigma$ corresponds to a Gaussian where the major axes of these ellipses are axis-aligned.

回忆一下,高斯模型的密度的形状是椭圆形的。对角矩阵$\Sigma$对应的就是椭圆的长轴(major axes)对齐(axis-aligned)的高斯模型。

有时候,我们还要对这个协方差矩阵(convariance matrix)给出进一步的约束,不仅设为对角的(major axes),还要求所有的对角元素(diagonal entries)都相等。这时候,就有$\Sigma=\sigma^2I$,其中$\sigma^2$是我们控制的参数。对于这个$\sigma^2$的最大似然估计则为:
$$
\sigma^2=\frac{1}{mn}\Sigma_{j=1}^{n}\Sigma_{i=1}^{m}(x_{j}^{i}-\mu_{j})^2
$$
这种模型对应的是密度函数为圆形轮廓的高斯模型(在二维空间也就是平面中是圆形,在更高维度当中就是球(spheres)或者超球体(hyperspheres))。
如果我么对数据要拟合一个完整的,不受约束的(unconstrained )协方差矩阵$\Sigma$,就必须满足$m\geq n+1$,这样才使得对$\Sigma$的最大似然估计不是奇异矩阵(singular matrix)。在上面提到的两个约束条件之下,只要$m\geq 2$,我们就能获得非奇异的(non-singular)$\Sigma$。

然而,讲$\Sigma$限定为对角矩阵,也就意味着对数据中不同坐标(coordinates)的$x_{i},x_{j}$建模都不相关(uncorrelated),且互相独立(independent)。通常,还是从样本数据里获得某些有趣的相关信息结构比较好。如果使用上面对$\Sigma$的某一种约束,就可能没有办法获取这些信息了。在本章讲义里面,我们会提到因子分析模型(factor analysis model),这个模型使用的参数比对角矩阵$\Sigma$更多,而且能从数据中获取某些相关性的信息(captures some correlations),但也不能对完整的协方差矩阵(full covariance matrix)进行拟合。
\end{spacing}
\subsection{多重高斯模型(Gaussians)的边界(Marginal)和条件(Conditional)}
\begin{spacing}{1.5}

\quad \quad 在讲解因子分析之前(factor analysis )之前,我们要说一下一个联合多元高斯分析(joint multivariate Gaussian distribution)下的随机变量(random variables)的条件(conditional)和边界(marginal分布(distributions)。

假如我们有一个值为向量的随机变量(vector-valued random variable):
$$
x=\begin{bmatrix}
x_1 \\ x_2
\end{bmatrix}
$$
其中$x_1\in R^{r},x_2\in R^{s}$,因此$x\in R^{r+s}$。设 $x\sim N(\mu ,\Sigma)$,即以$\mu$和$\Sigma$为参数的正太分布,则这两个参数为:
$$
\mu=\begin{bmatrix}
\mu_1\\\mu_2
\end{bmatrix}
,
\Sigma=\begin{bmatrix}
\Sigma_{11} &\Sigma_{12} \\ \Sigma_{21} &\Sigma_{22}
\end{bmatrix}
$$
其中,$\mu\in R^{r},\mu_2\in R^{s},\Sigma_{11}\in R^{r\times r},\Sigma_{12}^{r\times s}$ ,依次类推。由于协方差矩阵(convariance matrix)是对称的(symmetric),所以有$\Sigma_{12}=\Sigma_{21}^{T}$。

基于我们的假设,$x_1$和$x_2$是联合多元高斯分布(jointly multivariate Gaussian)。那么$x_1$的边界分布是什么? 不难看出$x_1$的期望$E[x_1]=\mu_1$,而协方差$Cov(x_1)=E[(x_1-\mu_1)(x_1-\mu_1)]=\Sigma_{11}$,接下来为了验证后面一项成立,要用$x_1$和$x_2$的联合方差的概念。
$$
\begin{aligned}
Cov(x)&=\Sigma \\
&=\begin{bmatrix}
\Sigma_{11} & \Sigma_{12} \\
\Sigma_{21} & \Sigma_{22}
\end{bmatrix} \\
&=E[(x-\mu)(x-u)^{T}]\\
&=E\begin{bmatrix}
\begin{pmatrix}
x_1-\mu_1 \\ x_2-\mu_2
\end{pmatrix}
\begin{pmatrix}
x_1-\mu_1 \\ x_2-\mu_2
\end{pmatrix}^{T}
\end{bmatrix}\\
&=E\begin{bmatrix}
\begin{pmatrix}
(x_1-\mu_1)(x_1-\mu_1)^T \\ (x_2-\mu_2)(x_1-\mu_1)^T
\end{pmatrix}
\begin{pmatrix}
(x_1-\mu_1)(x_2-\mu_2)^T \\ (x_2-\mu_2)(x_2-\mu_2)^{T}
\end{pmatrix}^{T}
\end{bmatrix}
\end{aligned}
$$

Matching the upper-left sub blocks in the matrices in the second and the last lines above gives the result.

高斯分布的边界条件(marginal distributions)本身也是高斯分布。所以我么就可以给出一个正太分布$x_1~N(\mu_1,\Sigma_{11})$来作为$x_1$的边界分布(marginal distributions)。

此外,我们还可以提出领一个问题,给定$x_2$的情况下$x_1$的条件分布是什么呢? 通过参考多元高斯分布的定义,就能得到条件分布$x_1|x_2~N(\mu_{1|2},\Sigma_{1|2})$为:
$$
\mu_{1|2}=\mu_1+\Sigma_{12}\Sigma_{22}^{-1}(x_2-\mu_2) \quad(1)\quad
\Sigma_{1|2}=\Sigma_{11}-\Sigma_{12}\Sigma_{22}^{-1}\Sigma_{21}\quad(2)
$$

在下一节对因子分析模型(factor analysis model)的讲解中,上面这些公式就很有用了,可以帮助高斯分布的条件和边界分布(conditional and marginal distributions)

\subsection{因子分析模型(Factor analysis model)}

在因子分析模型(Factor analysis model)中,我们定制在$(x,z)$上的一个联合分布,如下所示,其中$z\in R^{k}$是一个潜在随机变量(latent randon variable):
$$
\begin{aligned}
z~N(0,I)  \\
x|z~N(\mu+\Lambda z,\Psi)
\end{aligned}
$$

上面的式子中,我们这个模型中的参数是向量$\mu\in R^{n}$,矩阵$\Lambda\in R^{n\times k}$,以及一个对角矩阵$\Psi\in R^{n\times n}$。 $k$的值通常都选择比$n$小一点的。

这样,我们就设想每个数据点$x^{i}$都是通过一个$k$维度的多元高斯分布$z^{i}$中取样获得的。然后,通过计算$\mu+\Lambda z^{i}$,就可以映射到实数域$R^{n}$中的一个$k$维仿射空间$(k-dimensional affine space)$,在$\mu+\Lambda z^{(i)}$上加上协方差$\Psi$噪音,就得到了$x^{(i)}$。

反过来,咱们也就可以定义因子分析模型(factor anaylsis model),使用下面的设定:
$$
\begin{aligned}
& z~N(0,I)\\
& \xi~N(0,\psi)\\
& x=\mu+\Lambda z+\xi
& \end{aligned}
$$
其中的$\xi$和$z$是相互独立的。然后咱们来确切地看看这个模型定义的分布(distribution our)。其中,随机变量$z$和$x$有一个联合高斯分布(joint Gaussian distribution):
$$
\begin{bmatrix}
z\\x
\end{bmatrix} ~N(\mu_{zx},\Sigma)
$$
然后咱们要找到$\mu_{zx}$和$\Sigma$。

我们知道$z$的期望$E(z)=0$,这是因为$z$服从的是均值为0的正太分布$z~N(0,I)$。此外我们还知道:
$$
\begin{aligned}
E[x]&=E[\mu+\Lambda z+\xi]\\
&=\mu+\Lambda E[z]+E[\xi]\\
&=\mu
\end{aligned}
$$
综合以上这些条件,就得到了:
$$
\mu_{zx}=\begin{bmatrix}
\vec{0} \\ \mu
\end{bmatrix}
$$

下一步就是要找出$\Sigma$,我们需要计算出$\Sigma=E[x-E[Z](z-E(x))^{T}]$(矩阵$\Sigma$左上部分(upper-left block)),$\Sigma_{xx}=E[(z-E(z))(x-E(x))^{T}]$(右上部分(upper-right block)),以及$E[(x-E(x))(x-E(x))^{t}]$(右下部分(lowe-right block))。

由于$z$是一个正太分布$z~N(0,1)$,很容易就能知道$\Sigma_{zz}=Cov(z)=I$,另外:
$$
\begin{aligned}
E[(z-E(z))(x-E[x])^{T}]&=E[z(\mu+\Lambda z+\xi-\mu)^{T}]\\
&=E[zz^{T}]+E[z\xi^{T}]\\
&=\Lambda^{T}
\end{aligned}
$$

在上面的最后一步中,使用到了结论$E[zz^{T}]=Cov(z)$(因为z的均值为0,而且$E[z\xi]=E[z]E[\xi^{T}]$)(因为$z$和$\xi$相互独立,因此乘积(product)的期望(expectation)等于期望的乘积)。

同样的方法,我们可以用下面的方法论来找到$\Sigma_{xx}$:
$$
\begin{aligned}
E[(x-E(x))(x-E[x])^{T}]&=E[(\mu+\Lambda z+\xi-\mu)(\mu+\Lambda z+\xi-\mu)^{T}]\\
&=E[\Lambda zz^{T}\Lambda^{T}]+E[\xi z^{T}\Lambda^{T}+\Lambda z\xi^{T}+\xi\xi^{T}]\\
&=\Lambda E[zz^{T}]\Lambda^{T}+E[\xi\xi^{T}]\\
&=\Lambda\Lambda^{T}+\Psi
\end{aligned}
$$
把上面综合到一起,就得到了:
$$
\mu_{zx}=\begin{bmatrix}
z \\ hx
\end{bmatrix}
~N\begin{pmatrix}
\begin{bmatrix}\vec{0}\\ \mu\end{bmatrix},
\begin{bmatrix}I&\Lambda^{T}\\ \Lambda &\Lambda\Lambda^{T}+\Psi\end{bmatrix}
\end{pmatrix}
$$
因此,我们还能发现$x$边界分布(marginal distribution)为:$x~N(\mu,\Lambda\Lambda^{T}+\Psi)$。所以,给定一个训练样本集合$\{x^{(i);i=1\cdots,m}\}$,参数(parameters)的最大似然估计函数的对数函数(log likelihood),就可以写为:
$$
l(\mu,\Lambda,\Psi)=log\prod_{i=1}^{m}\frac{1}{(2\pi)^{n\over 2}|\Lambda\Lambda^{T}+\Psi|^{\frac{1}{2}}}exp(-\frac{1}{2}(x^{(i)-\mu})^{T}(\Lambda\Lambda^{T}+\Psi)^{-1}(x^{i}-\mu))
$$

为了进最大似然估计,我们就要最大化上面这个参数的函数。但确切地对上面这个方程进行最大化,是很艰难的,不信你自己试试哈,而且我们都知道没有算法能够以封闭形式(closed-form)来实现对大化。所以,我们就该用期望最大化算法(EM algorithm)。下一节里面,咱们就来推导一下针对因子分析模型(factor analysis)的期望最大化算法(EM)
\subsection{针对因子分析模型(factor analysis)的期望最大化算法(EM))}
EM算法步骤的推导很简单。只需要计算出来$Q_{i}(z^{i})=p(z^{(i)}|x^{(i)};\mu,\Lambda,\Psi)$。把等式$(3)$当中给出的分布带入到方程$(1-2)$,来找出一个高斯分布的条件分布,我们就能发现$z^{(i)|x^{(i)}};\mu,\Lambda,\Psi \sim N(\mu_{z^{i}},\Sigma_{z^{(i)}|x^{i}})$,其中:
$$
\begin{aligned}
\mu_{z^{(i)|x^{(i)}}}&=\Lambda^{T}(\Lambda\Lambda^{T}+\Psi)^{-1}(x^{(i)-\mu})\\
\Sigma_{z^{(i)}|x^{{(i)}}}&=I-\Lambda^{T}(\Lambda\Lambda^{T}+\Psi)^{-1}\Lambda
\end{aligned}
$$
所以通过对$\mu_{z^{(i)}|x^{i}}$和$\Sigma_{z^{(i)}|x^{i}}$,进行这样的定义,就能得到:
$$
Q_{i}(z^{i})=\frac{1}{(2\pi)^{k\over 2}|\Sigma_{z^{(i)}|x^{(i)}}|^{\frac{1}{2}}}exp(\frac{-1}{2}(z^{(i)}-\mu_{z^{(i)}|x^{(i)}})^{T}\Sigma_{z^{(i)}|x^{(i)}}^{-1}(z^{(i)}-\mu_{z^{(i)}|x^{(i)}})
$$

接下来就是$ M $步骤了。这里需要去最大化下面这个参数$\mu,\Lambda,\Psi $的函数值:
$$
\sum_{i=1}^{m}\int_{z^{(i)}}Q_{i}(z^{(i)})log\frac{p(x^{(i)},z^{(i)};\mu,\Lambda,\Psi)}{Q_{i}(z^{(i)})}dz^{(i)}\quad (4)
$$
我们在本文中仅仅对$\Lambda$进行优化,关于$\mu$和$\Psi$的更新就作为练习留给自己进行推导了。

把等式$(4)$简化成下面的形式:
$$
\begin{aligned}
\sum_{i=1}^{m}\int_{z^{(i)}}[log p(x^{(i)}|z^{(i)};\mu,\Lambda,\Psi)+logp(z^{(i)})-logQ_i(z^{(i)})]dz^{(i)} \quad (5) \\
=\sum_{i=1}^{m}E_{z^{(i)}~Q_i}[log p(x^{(i)}|z^{(i)};\mu,\Lambda,\Psi)+log p(z^{(i)})-log Q_{i}(z^{(i)})] \quad(6)
\end{aligned}
$$
上面的等式中,$z^{(i)}~Q_{i}$这个下标(subscript),表示的意思是这个期望是从$Q_{i}$中取得$z^{i}$的。在后续的推导过程中,如果没有歧义的情况下,我们就会把这个省略掉。删除这些不依赖参数的项目后,我们就发现只需要最大化:
$$$$
我们对上面的函数进行关于$\Lambda$的最大化。可见只有最后的一项依赖$\Lambda$。求导数,同时利用下面几个结论:$Tr a=a(for a \in R),Tr AB=Tr BA,\nabla_{A}Tr ABA^{T}C=CAB+C^{T}AB$,就能得到:
$$
\begin{aligned}
 &\nabla_{\lambda}\sum_{i=1}^{m}-E[\frac{1}{2}(x^{i}-\mu-\Lambda z^{i})^{T}\Psi^{-1}(x^{i}-\mu-\Lambda z^{i})]\\
 &=\sum_{i=1}^{m}\nabla_{\Lambda}E[-tr\frac{1}{2}z^{i^{T}}\Lambda^{T}\Psi^{-1}\Lambda z^{i}+trz^{i^{T}}\Lambda ^{T}\Psi^{-1}(x^{i}-\mu)]\\
 &=\sum_{i=1}^{m}\nabla_{\Lambda}E[-tr\frac{1}{2}\Lambda^{T}\Psi^{-1}\Lambda z^{i}z^{i^{T}}+tr\Lambda ^{T}\Psi^{-1}(x^{i}-\mu)z^{i^{T}}]\\
 &=\sum_{i=1}^{m}E[-\Psi^{-1}\Lambda z^{i}z^{i^{T}}+\Psi^{-1}(x^{i}-\mu)z^{i^{T}}]
\end{aligned}
$$
设置导数为$0$,然后简化,就能得到:
$$
\sum_{i=1}^{m}\Lambda E_{z^{i}~Q_{i}}[z^{i}z^{i^{T}}]=\sum_{i=1}^{m}(x^{i}-\mu)E_{z^{i}~Q_{i}}[z^{i^{T}}]
$$
接下来,求解$\Lambda$,就能得到:
$$\Lambda=(\sum_{i=1}^{m}(x^{i}-\mu)E_{z^{i}~Q_{i}}[z^{i^{T}}])(\sum_{i=1}^{m}E_{z^{i}~Q_{i}}[z^{i}z^{i^{T}}])^{-1} \quad(7)$$

有一个很有意思的地方需要注意,上面这个等式和用最小二乘线性回归推出的正则方程有密切的关系:
$$\theta^{T}=(y^{T}X)(X^{T}X)^{-1}$$

与之类似,这里的$x$是一个关于$z$(以及噪声noise)的线性方程。考虑在$E$步骤中对$z$已经给出猜测,接下来就可以来尝试与$x$和$z$相关的位置量$\Lambda$进行估计。接下来不出意料,我们就会得到某种类似正则方程的结果。然而,这个还是和利用对$z$的最佳猜测(best guesses)进行最下二乘算法有一个很大的区别的;这一点我们很快就会看到。

为了完成$M$步骤的更新,接下来我们要解出等式$(7)$当中的期望值(values of the expectations)。由于我们定义$Q_{i}$是均值为$\mu_{z^{i}|x^{i}}$,协方差为$\\Sigma_{z^{i}|x^{i}}$的一个高斯分布,所以很容易得到:
$$
\begin{aligned}
E_{z^{i}~Q_{i}}[z^{i^{T}}]&=\mu^{T}_{z^{i}|x^{i}}\\
E_{z^{i}~Q_{i}}[z^{i}z^{i^{T}}]&=\mu_{z^{i}|x^{i}}\mu^{T}_{z^{i}|x^{i}}+\Sigma_{z^{i}|x^{i}}
\end{aligned}
$$

上面第二个等式的推导依赖与下面这个事实:对于一个随机变量$Y$,协方差$Cov(Y)=E[YY^{T}]-E[Y]E[Y]^{T}$,所以$E[YY^{T}]=E[Y]E[Y]^T+Cov(Y)$。把这个带入道等式$(7)$,就得到$M$步骤中$\Lambda$的更行规则:
$$
\Lambda=(\sum_{i=1}^{m}(x^{i}-\mu)\mu^{T}_{z^{i}|x^{i}})(\sum_{i=1}^{m}\mu_{z^{i}|x^{i}}\mu_{z^{i}|x^{i}}^{T}+\Sigma_{z^{i}|x^{i}})^{-1} \quad (8)
$$

上面这个等式中,要特别注意等号右边这一侧的$\Sigma_{z^{i}|x^{i}}$。这是一个根据
$z^{i}$给出的$x^{i}$后验发分布$p(z^{i}|x^{i})$的协方差,而在M步骤中必须考虑到在这个后验分布中$z^{i}$的不确定性。推导EM算法的一个常见错误就是在E步骤进行假设,只需要算出在随机变量$z$的期望$E[z]$,然后把这个值放到M步骤当中$z$出现的每个地方进行优化。当然,这能解决简单问题,例如高斯混合模型,在因子模型的推导过程中,就同时需要$E[zz^{T}]$和$E[z]$;而我们已经知道,$E[zz^{T}]$和$E[z]E[z]^{T}$随着$\Sigma_{z|x}$而变化。因此,在M步骤就必须要考虑到后验分布$p(z^{i}|x^{i})$中$z$的协方差。

最后,我们还可以发现,在M步骤对参数$\mu$和$\Psi$的优化。不难发现其中的$\mu$为:
$$
\mu=\frac{1}{m}\sum_{i=1}^{m}x^{i}
$$

由于这个值随着参数的变换而该改变(也就是说,和$\Lambda$的更新不同,这里等式右侧不依赖$Q_{i}(z^{i})=p(z^{i}|x^{i};\mu,\Lambda,\Psi)$),这个$Q_i(z^{i})$是依赖参数的,这个只需要计算一次就可以,在算法运行过程中,也不需要进一步更新。类似地,对角矩阵$\Psi$也可以通过计算下面这个式子来获得:
$$
\phi=\frac{1}{m}\sum_{i=1}^{m}x^{i}x^{i^{T}}-x^{i}\mu_{z^{i}|x^{i}}^T\Lambda^{T}-\Lambda\mu_{z^{i}|x^{i}}x^{i^{T}}+\Lambda(\mu_{z^{i}|x^{i}}\mu_{z^{i}|x^{i}}^{T}+\Sigma_{z^{i}|x^{i}})\Lambda^{T}
$$

然后只需要设$\Psi_{ii}=\Phi_{ii}$(也就是说,设$\Psi$为一个仅仅包含矩阵$\Phi$中对角线元素的对角矩阵)
\end{spacing}


\end{document}